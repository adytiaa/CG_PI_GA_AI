\documentclass[11pt]{article}
\usepackage{amsmath,amssymb,amsfonts}
\usepackage{bm}
\usepackage{geometry}
\usepackage{physics}
\usepackage{hyperref}
\geometry{margin=1in}

\title{Improved Physics-Informed Neural Network for Thermal--Fluid Coupling in Czochralski Silicon Crystal Growth}
\author{}
\date{}

\begin{document}
\maketitle

\section{Introduction}

This document summarizes the governing equations, modeling assumptions, training methodology,
and mathematical structure of the \emph{improved physics-informed neural network (PINN)}
proposed in:

\begin{quote}
Research on the thermal--fluid coupling in the growth process of Czochralski silicon single crystals
based on an improved physics-informed neural network,\\
\emph{AIP Advances}, 15, 105202 (2025).
\end{quote}

The model targets steady, incompressible, thermally driven melt flow during Czochralski (CZ)
silicon crystal growth and introduces spatial-information enhancement and adaptive loss balancing.

\section{Governing Equations}

The melt is modeled as a steady, incompressible Newtonian fluid under the Boussinesq approximation.
A two-dimensional axisymmetric domain is assumed.

\subsection{Continuity Equation}

\begin{equation}
\nabla \cdot \bm{u} = 0
\end{equation}

In Cartesian form (as used in the paper's 2D numerical implementation):
\begin{equation}
\frac{\partial u}{\partial x} + \frac{\partial v}{\partial y} = 0
\end{equation}

\subsection{Momentum Equations}

The steady incompressible Navier--Stokes equations with buoyancy coupling are given by
\begin{equation}
\rho (\bm{u} \cdot \nabla)\bm{u}
= -\nabla p + \mu \nabla^2 \bm{u}
+ \rho \bm{g}\beta (T - T_0).
\end{equation}

Component-wise:
\begin{align}
u \frac{\partial u}{\partial x} + v \frac{\partial u}{\partial y}
&= -\frac{1}{\rho} \frac{\partial p}{\partial x}
+ \nu \left( \frac{\partial^2 u}{\partial x^2}
+ \frac{\partial^2 u}{\partial y^2} \right), \\
u \frac{\partial v}{\partial x} + v \frac{\partial v}{\partial y}
&= -\frac{1}{\rho} \frac{\partial p}{\partial y}
+ \nu \left( \frac{\partial^2 v}{\partial x^2}
+ \frac{\partial^2 v}{\partial y^2} \right)
+ g\beta(T - T_0).
\end{align}

\subsection{Energy Equation}

\begin{equation}
\rho c_p (\bm{u} \cdot \nabla T)
= k \nabla^2 T
\end{equation}

or equivalently,
\begin{equation}
u \frac{\partial T}{\partial x} + v \frac{\partial T}{\partial y}
= \alpha \left(
\frac{\partial^2 T}{\partial x^2}
+ \frac{\partial^2 T}{\partial y^2}
\right),
\end{equation}
where $\alpha = k / (\rho c_p)$.

\section{Modeling Assumptions}

\begin{itemize}
\item \textbf{Incompressibility}: density is constant except in the buoyancy term.
\item \textbf{Boussinesq approximation}: thermal expansion drives buoyancy.
\item \textbf{Laminar flow}: turbulence is neglected.
\item \textbf{Steady-state}: time derivatives are omitted.
\item \textbf{Axisymmetry}: azimuthal velocity is neglected.
\item \textbf{Constant properties}: $\nu$, $k$, $c_p$ are constant.
\end{itemize}

\section{Dimensionless Formulation}

Let characteristic scales be:
\[
L, \quad U, \quad \Delta T.
\]

Define dimensionless variables:
\begin{equation}
x^* = \frac{x}{L}, \quad
\bm{u}^* = \frac{\bm{u}}{U}, \quad
T^* = \frac{T - T_0}{\Delta T}, \quad
p^* = \frac{p}{\rho U^2}.
\end{equation}

The dimensionless equations become:

\subsection{Continuity}
\begin{equation}
\nabla^* \cdot \bm{u}^* = 0
\end{equation}

\subsection{Momentum}
\begin{equation}
(\bm{u}^* \cdot \nabla^*) \bm{u}^*
= -\nabla^* p^*
+ \frac{1}{\mathrm{Re}} \nabla^{*2} \bm{u}^*
+ \mathrm{Ra}\,\mathrm{Pr}^{-1} T^* \bm{e}_g
\end{equation}

\subsection{Energy}
\begin{equation}
(\bm{u}^* \cdot \nabla^*) T^*
= \frac{1}{\mathrm{Re}\,\mathrm{Pr}} \nabla^{*2} T^*
\end{equation}

where
\begin{align}
\mathrm{Re} &= \frac{UL}{\nu}, \\
\mathrm{Pr} &= \frac{\nu}{\alpha}, \\
\mathrm{Ra} &= \frac{g\beta \Delta T L^3}{\nu \alpha}.
\end{align}

\section{PINN Approximation}

The neural network approximates:
\begin{equation}
\mathcal{N}_\theta(x,y) \rightarrow \{u,v,p,T\}.
\end{equation}

Automatic differentiation is used to compute all spatial derivatives.

\section{Improved PINN vs Standard PINN}

\subsection{Standard PINN}

A standard PINN minimizes:
\begin{equation}
\mathcal{L}_{\text{std}} =
\sum_i \| \mathcal{R}_i \|^2
+ \sum_j \| \mathcal{B}_j \|^2,
\end{equation}
with fixed loss weights and spatial coordinates only entering at the input layer.

\subsection{Improved PINN (SI--LB PINN)}

The improved PINN introduces:

\paragraph{Spatial Information Embedding}
\begin{equation}
h_\ell = \sigma\left(
W_\ell h_{\ell-1}
+ Z_\ell \odot M(x,y)
+ (1-Z_\ell) \odot N(x,y)
\right),
\end{equation}
which preserves geometric information in deep layers.

\paragraph{Adaptive Loss Balancing}
\begin{equation}
\mathcal{L}
= \sum_m \left(
e^{-\log\sigma_m^2} \mathcal{L}_m + \log\sigma_m
\right),
\end{equation}
where $\sigma_m$ are trainable parameters.

This improves conditioning for multi-physics coupling.

\section{Mapping Governing Equations to PyTorch Code}

\subsection{Continuity Residual}
\begin{verbatim}
u_x = autograd.grad(u, x, grad_outputs=ones, create_graph=True)[0]
v_y = autograd.grad(v, y, grad_outputs=ones, create_graph=True)[0]
R_cont = u_x + v_y
\end{verbatim}

\subsection{Momentum Residual (x-direction)}
\begin{verbatim}
u_xx = autograd.grad(u_x, x, grad_outputs=ones, create_graph=True)[0]
u_yy = autograd.grad(u_y, y, grad_outputs=ones, create_graph=True)[0]
p_x  = autograd.grad(p, x, grad_outputs=ones, create_graph=True)[0]

R_mom_x = u*u_x + v*u_y + p_x - nu*(u_xx + u_yy)
\end{verbatim}

\subsection{Momentum Residual (y-direction)}
\begin{verbatim}
p_y = autograd.grad(p, y, grad_outputs=ones, create_graph=True)[0]

R_mom_y = u*v_x + v*v_y + p_y - nu*(v_xx + v_yy) - g*beta*(T-T0)
\end{verbatim}

\subsection{Energy Residual}
\begin{verbatim}
T_xx = autograd.grad(T_x, x, grad_outputs=ones, create_graph=True)[0]
T_yy = autograd.grad(T_y, y, grad_outputs=ones, create_graph=True)[0]

R_energy = u*T_x + v*T_y - alpha*(T_xx + T_yy)
\end{verbatim}

\section{Training Process}

\begin{itemize}
\item Collocation points sampled inside the domain
\item Boundary condition points enforced via penalty loss
\item Adam optimizer for training
\item No labeled CFD or experimental data required
\item COMSOL solutions used only for validation
\end{itemize}

\section{Axisymmetric Cylindrical Coordinate Formulation}

The Czochralski crystal growth process is inherently axisymmetric.
We therefore adopt cylindrical coordinates $(r,z,\theta)$ and assume
$\partial / \partial \theta = 0$ and $u_\theta = 0$.

The velocity field is:
\[
\bm{u} = (u_r(r,z),\, u_z(r,z))
\]

All governing equations include geometric source terms arising from the cylindrical coordinate system.

---

\subsection{Continuity Equation}

The incompressible continuity equation in axisymmetric coordinates is:

\begin{equation}
\frac{1}{r}\frac{\partial}{\partial r}(r u_r)
+ \frac{\partial u_z}{\partial z}
= 0
\end{equation}

This term is critical and is a common source of error in naïve PINN implementations.

---

\subsection{Momentum Equations}

\subsubsection{Radial Momentum}

\begin{equation}
u_r \frac{\partial u_r}{\partial r}
+ u_z \frac{\partial u_r}{\partial z}
=
-\frac{1}{\rho} \frac{\partial p}{\partial r}
+ \nu
\left(
\frac{1}{r}\frac{\partial}{\partial r}
\left(r \frac{\partial u_r}{\partial r}\right)
+ \frac{\partial^2 u_r}{\partial z^2}
- \frac{u_r}{r^2}
\right)
\end{equation}

---

\subsubsection{Axial Momentum}

\begin{equation}
u_r \frac{\partial u_z}{\partial r}
+ u_z \frac{\partial u_z}{\partial z}
=
-\frac{1}{\rho} \frac{\partial p}{\partial z}
+ \nu
\left(
\frac{1}{r}\frac{\partial}{\partial r}
\left(r \frac{\partial u_z}{\partial r}\right)
+ \frac{\partial^2 u_z}{\partial z^2}
\right)
+ g\beta (T - T_0)
\end{equation}

---

\subsection{Energy Equation}

The heat transport equation in axisymmetric form is:

\begin{equation}
u_r \frac{\partial T}{\partial r}
+ u_z \frac{\partial T}{\partial z}
=
\alpha
\left(
\frac{1}{r}\frac{\partial}{\partial r}
\left(r \frac{\partial T}{\partial r}\right)
+ \frac{\partial^2 T}{\partial z^2}
\right)
\end{equation}

---

\section{Dimensionless Axisymmetric Form}

Using characteristic scales $(L, U, \Delta T)$:

\[
r^* = \frac{r}{L}, \quad
z^* = \frac{z}{L}, \quad
u^* = \frac{u}{U}, \quad
T^* = \frac{T - T_0}{\Delta T}
\]

The dimensionless equations become:

---

\subsection{Continuity}

\begin{equation}
\frac{1}{r^*}\frac{\partial}{\partial r^*}
(r^* u_r^*)
+ \frac{\partial u_z^*}{\partial z^*}
= 0
\end{equation}

---

\subsection{Momentum}

\begin{align}
u_r^* \partial_{r^*} u_r^*
+ u_z^* \partial_{z^*} u_r^*
&=
-\partial_{r^*} p^*
+ \frac{1}{\mathrm{Re}}
\left(
\frac{1}{r^*}\partial_{r^*}(r^* \partial_{r^*} u_r^*)
+ \partial_{z^*z^*} u_r^*
- \frac{u_r^*}{r^{*2}}
\right), \\
u_r^* \partial_{r^*} u_z^*
+ u_z^* \partial_{z^*} u_z^*
&=
-\partial_{z^*} p^*
+ \frac{1}{\mathrm{Re}}
\left(
\frac{1}{r^*}\partial_{r^*}(r^* \partial_{r^*} u_z^*)
+ \partial_{z^*z^*} u_z^*
\right)
+ \frac{\mathrm{Ra}}{\mathrm{Re}\,\mathrm{Pr}} T^*
\end{align}

---

\subsection{Energy}

\begin{equation}
u_r^* \partial_{r^*} T^*
+ u_z^* \partial_{z^*} T^*
=
\frac{1}{\mathrm{Re}\,\mathrm{Pr}}
\left(
\frac{1}{r^*}\partial_{r^*}(r^* \partial_{r^*} T^*)
+ \partial_{z^*z^*} T^*
\right)
\end{equation}

---

\section{Weak Formulation (COMSOL-Consistent)}

COMSOL solves the \textbf{weak (variational) form} of the governing equations.
Let $v_r, v_z, q, w$ be admissible test functions.

---

\subsection{Continuity Weak Form}

\begin{equation}
\int_\Omega
\left(
\frac{1}{r}\frac{\partial}{\partial r}(r u_r)
+ \frac{\partial u_z}{\partial z}
\right) q \, r\,dr\,dz
= 0
\end{equation}

Note the \textbf{Jacobian factor $r$}, which is essential in axisymmetric COMSOL models.

---

\subsection{Radial Momentum Weak Form}

\begin{align}
\int_\Omega
\left[
(u_r \partial_r u_r + u_z \partial_z u_r)v_r
- p \partial_r v_r
+ \nu (\nabla u_r \cdot \nabla v_r)
- \nu \frac{u_r}{r^2} v_r
\right]
r \, dr\,dz
= 0
\end{align}

---

\subsection{Axial Momentum Weak Form}

\begin{align}
\int_\Omega
\left[
(u_r \partial_r u_z + u_z \partial_z u_z)v_z
- p \partial_z v_z
+ \nu (\nabla u_z \cdot \nabla v_z)
+ g\beta (T - T_0) v_z
\right]
r \, dr\,dz
= 0
\end{align}

---

\subsection{Energy Weak Form}

\begin{equation}
\int_\Omega
\left[
(u_r \partial_r T + u_z \partial_z T) w
+ \alpha (\nabla T \cdot \nabla w)
\right]
r \, dr\,dz
= 0
\end{equation}

---

\section{PINN Residuals vs COMSOL Weak Residuals}

\paragraph{Key Correspondence}

\begin{itemize}
\item \textbf{PINN}: Enforces \emph{strong-form residuals} at collocation points
\item \textbf{COMSOL}: Enforces \emph{weak-form residuals} over finite elements
\end{itemize}

Mathematically:
\[
\mathcal{R}_{\text{PINN}}(x_i) \approx 0
\quad \Longleftrightarrow \quad
\int_\Omega \mathcal{R}_{\text{weak}} \, d\Omega = 0
\]

XPINN partially bridges this gap by enforcing local subdomain consistency,
which mimics element-wise weak enforcement.

---

\section{Mapping to PyTorch Residuals (Axisymmetric)}

\begin{verbatim}
# Continuity
R_cont = (u_r + r*u_r_r)/r + u_z_z

# Radial momentum
R_ur = u_r*u_r_r + u_z*u_r_z + p_r \
       - nu*((u_r_r + r*u_r_rr)/r + u_r_zz - u_r/r**2)

# Axial momentum
R_uz = u_r*u_z_r + u_z*u_z_z + p_z \
       - nu*((u_z_r + r*u_z_rr)/r + u_z_zz) \
       - g*beta*(T - T0)

# Energy
R_T = u_r*T_r + u_z*T_z \
      - alpha*((T_r + r*T_rr)/r + T_zz)
\end{verbatim}

---

\section{Remarks}

\begin{itemize}
\item Axisymmetric geometric terms are essential for physical accuracy
\item COMSOL weak-form includes the Jacobian factor $r$
\item PINNs approximate the strong form but converge to the same solution
\item XPINNs improve conditioning by mimicking FEM locality
\end{itemize}


\section{Conclusion}

The improved SI--LB PINN provides a mathematically well-conditioned framework
for solving coupled thermal--fluid PDEs in Czochralski crystal growth,
significantly improving convergence and accuracy over standard PINNs.

\end{document}


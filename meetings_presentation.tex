 %%%%%%%%%%%%%%%%%%%%%%%%%%%%%%%%%%%%%%%%%
% Beamer Presentation
% LaTeX Template
% Version 1.0 (10/11/12)
%
% This template has been downloaded from:
% http://www.LaTeXTemplates.com
%
% License:
% CC BY-NC-SA 3.0 (http://creativecommons.org/licenses/by-nc-sa/3.0/)
%
%%%%%%%%%%%%%%%%%%%%%%%%%%%%%%%%%%%%%%%%%

%----------------------------------------------------------------------------------------
%	PACKAGES AND THEMES
%----------------------------------------------------------------------------------------

\documentclass{beamer}

\mode<presentation> {


% The Beamer class comes with a number of default slide themes
% which change the colors and layouts of slides. Below this is a list 
% of all the themes, uncomment each in turn to see what they look like.

%\usetheme{default}
%\usetheme{AnnArbor}
%\usetheme{Antibes}
%\usetheme{Bergen}
%\usetheme{Berkeley}
%\usetheme{Berlin}
%\usetheme{Boadilla}
%\usetheme{CambridgeUS}
%\usetheme{Copenhagen}
%\usetheme{Darmstadt}
%\usetheme{Dresden}
%\usetheme{Frankfurt}
%\usetheme{Goettingen}
%\usetheme{Hannover}
%\usetheme{Ilmenau}
%\usetheme{JuanLesPins}
%\usetheme{Luebeck}
\usetheme{Madrid}
%\usetheme{Malmoe}
%\usetheme{Marburg}
%\usetheme{Montpellier}
%\usetheme{PaloAlto}
%\usetheme{Pittsburgh}
%\usetheme{Rochester}
%\usetheme{Singapore}
%\usetheme{Szeged}
%\usetheme{Warsaw}

% As well as themes, the Beamer class has a number of color themes
% for any slide theme. Uncomment each of these in turn to see how it
% changes the colors of your current slide theme.

%\usecolortheme{albatross}
%\usecolortheme{beaver}
%\usecolortheme{beetle}
%\usecolortheme{crane}
%\usecolortheme{dolphin}
%\usecolortheme{dove}
%\usecolortheme{fly}
%\usecolortheme{lily}
%\usecolortheme{orchid}
%\usecolortheme{rose}
%\usecolortheme{seagull}
%\usecolortheme{seahorse}
%\usecolortheme{whale}
%\usecolortheme{wolverine}

%\setbeamertemplate{footline} % To remove the footer line in all slides uncomment this line
%\setbeamertemplate{footline}[page number] % To replace the footer line in all slides with a simple slide count uncomment this line

\setbeamertemplate{navigation symbols}{} % To remove the navigation symbols from the bottom of all slides uncomment this line
}

\usepackage{graphicx} % Allows including images
\usepackage{booktabs} % Allows the use of \toprule, \midrule and \bottomrule in tables
\usepackage{amsmath}
\usepackage{xcolor}
\usepackage[footnotesize]{subfigure}
\definecolor{olive}{rgb}{0.3, 0.4, .1}
\definecolor{fore}{RGB}{249,242,215}
\definecolor{back}{RGB}{51,51,51}
\definecolor{title}{RGB}{255,0,90}
\definecolor{dgreen}{rgb}{0.,0.6,0.}
\definecolor{gold}{rgb}{1.,0.84,0.}
\definecolor{Red}{cmyk}{0.99,0,0.52,0}
\definecolor{Red}{cmyk}{0.85,0,0.33,0}
\definecolor{RawSienna}{cmyk}{0,0.72,1,0.45}
\definecolor{Magenta}{cmyk}{0,1,0,0}
\newcommand{\farbig}[2][red]{\textcolor{#1}{#2}}
\newcommand{\mysize}[3][1.2]{%
  \set@fontsize\baselinestretch{#2}{#2}%
  \set@fontsize{#1}\f@size\f@baselineskip%
  {\selectfont#3}
}
\newcommand{\vect}[1]{\textbf{\textit{#1}}}
\renewcommand{\vec}[1]{\mbox{\boldmath \small $#1$}}
\newcommand{\AT}{{\textrm{{AT}}}}
\newcommand{\EX}{{\textrm{EX}}}
\newcommand{\CG}{{\textrm{CG}}}
\newcommand{\HY}{{\Delta}}
%----------------------------------------------------------------------------------------
%	TITLE PAGE
%----------------------------------------------------------------------------------------
\title{physics-AI For CFD Simulations: SciML} % The short title appears at the bottom of every slide, the full title is only on the title page

\author{} % Your name

\date{\today} % Date, can be changed to a custom date

\begin{document}

\begin{frame}
\titlepage % Print the title page as the first slide
\end{frame}

\begin{frame}
\frametitle{Outline} % Table of contents slide, comment this block out to remove it
\tableofcontents % Throughout your presentation, if you choose to use \subsection{} and \subsection{} commands, these will automatically be printed on this slide as an overview of your presentation
\end{frame}

%----------------------------------------------------------------------------------------
%	PRESENTATION SLIDES
%----------------------------------------------------------------------------------------


\section{Adapted Physics-Informed Neural Networks for Crystal Growth}
\section{Modeling of Cryogenic Systems-H2, N2}
\section{Dyad Agent- Jüiahüb}

\subsection{Cryogenic H2/N2}

\begin{frame}
\begin{figure}
\centering
%\includegraphics[width=0.5\textwidth]{Images/docker_events.png}
\includegraphics[width=0.8\textwidth]{cryo_1.jpg}
\end{figure}
\end{frame}


\subsection{Dyad agent}

\begin{frame}
\begin{figure}
\centering
%\includegraphics[width=0.5\textwidth]{Images/docker_events.png}
\includegraphics[width=0.8\textwidth]{Dyad.png}
\end{figure}
\end{frame}
%\subsection{Adapted PINNs for Crystal Growth}
%\subsection{PINNs for Cryogenic Systems}
%\subsection{Dyad Agent Framework}

\subsection{Czochralski Crystal Growth Process}

\begin{frame}{What is the Czochralski (CZ) Process?}
\begin{itemize}
\item Dominant method for single-crystal silicon growth
\item Used in semiconductor and photovoltaic industries
\end{itemize}
\end{frame}

\begin{frame}{CZ Furnace Schematic}
\placeholder{CZ Furnace: Crucible, Melt, Heater, Crystal Pulling}
\end{frame}

\begin{frame}{Crystal Growth Physics}
\begin{itemize}
\item Solidification at melt--crystal interface
\item Strong heat extraction through crystal
\item Melt flow affects dopant and defect transport
\end{itemize}
\end{frame}

\begin{frame}{Why Modeling is Difficult}
\begin{itemize}
\item Strong thermal--fluid coupling
\item High Rayleigh number convection
\item Thin boundary layers
\item Multi-physics and multi-scale nature
\end{itemize}
\end{frame}

%=====================================================
\subsection{Modeling Assumptions and Domain}

\begin{frame}{Axisymmetric Approximation}
\begin{itemize}
\item Nearly rotationally symmetric geometry
\item Reduces 3D problem to 2D $(r,z)$
\end{itemize}
\end{frame}

\begin{frame}{Computational Domain}
\placeholder{Axisymmetric Melt Domain in $(r,z)$}
\end{frame}

\begin{frame}{Boundary Identification}
\placeholder{Crucible Wall, Free Surface, Crystal Interface}
\end{frame}

\begin{frame}{Physical Assumptions}
\begin{itemize}
\item Incompressible Newtonian melt
\item Laminar, steady-state flow
\item Constant material properties
\item Boussinesq approximation
\end{itemize}
\end{frame}

%=====================================================
\subsection{Governing Equations}

\begin{frame}{Continuity Equation}
\[
\frac{1}{r}\frac{\partial}{\partial r}(r u_r)
+ \frac{\partial u_z}{\partial z} = 0
\]
\end{frame}

\begin{frame}{Radial Momentum Equation}
\[
u_r \partial_r u_r + u_z \partial_z u_r =
-\frac{1}{\rho}\partial_r p
+ \nu\left(\nabla^2 u_r - \frac{u_r}{r^2}\right)
\]
\end{frame}

\begin{frame}{Axial Momentum Equation}
\[
u_r \partial_r u_z + u_z \partial_z u_z =
-\frac{1}{\rho}\partial_z p
+ \nu\nabla^2 u_z
+ g\beta(T-T_0)
\]
\end{frame}

\begin{frame}{Energy Equation}
\[
u_r \partial_r T + u_z \partial_z T =
\alpha \nabla^2 T
\]
\end{frame}

\begin{frame}{Thermal--Fluid Coupling}
\begin{itemize}
\item Temperature $\rightarrow$ buoyancy force
\item Velocity $\rightarrow$ heat advection
\item Strong nonlinear feedback loop
\end{itemize}
\end{frame}

%=====================================================
\subsection{Dimensionless Analysis}

\begin{frame}{Rayleigh Number}
\[
\mathrm{Ra} = \frac{g\beta \Delta T L^3}{\nu \alpha}
\]
\begin{itemize}
\item Measures strength of natural convection
\end{itemize}
\end{frame}

\begin{frame}{High Rayleigh Number Regime}
\begin{itemize}
\item Thin thermal and velocity boundary layers
\item Strong nonlinear advection
\item Major source of instability for PINNs
\end{itemize}
\end{frame}

%=====================================================
\subsection{Improved PINN Methodology}

\begin{frame}{Standard PINN Concept}
\begin{itemize}
\item Neural network approximates PDE solution
\item Physics enforced via residual minimization
\end{itemize}
\end{frame}

\begin{frame}{PINN Inputs and Outputs}
\begin{itemize}
\item Inputs: $(r,z)$
\item Outputs: $u_r, u_z, p, T$
\end{itemize}
\end{frame}

\begin{frame}{Spatial Information (SI) Embedding}
\begin{itemize}
\item Coordinates injected into hidden layers
\item Preserves geometric sensitivity
\item Improves boundary-layer resolution
\end{itemize}
\end{frame}

\begin{frame}{Adaptive Loss Balancing}
\begin{itemize}
\item Separate losses for each equation
\item Trainable weights
\item Prevents gradient domination
\end{itemize}
\end{frame}

%=====================================================
\subsection{Equation-to-Loss Function Mapping}

\begin{frame}{PINN Loss Function Structure}
\[
\mathcal{L}
= \lambda_c \mathcal{L}_c
+ \lambda_m \mathcal{L}_m
+ \lambda_e \mathcal{L}_e
+ \lambda_b \mathcal{L}_b
\]
\end{frame}

\begin{frame}{Continuity Loss}
\[
\mathcal{L}_c =
\frac{1}{N}\sum_i
\left|
\frac{1}{r}\partial_r(r u_r)
+ \partial_z u_z
\right|^2
\]
\end{frame}

\begin{frame}{Momentum Loss}
\[
\mathcal{L}_m =
\|\mathcal{R}_{u_r}\|^2 + \|\mathcal{R}_{u_z}\|^2
\]
\begin{itemize}
\item Enforces Navier--Stokes equations
\end{itemize}
\end{frame}

\begin{frame}{Energy Loss}
\[
\mathcal{L}_e =
\frac{1}{N}\sum_i
\left|
u_r \partial_r T + u_z \partial_z T - \alpha\nabla^2 T
\right|^2
\]
\end{frame}

\begin{frame}{Boundary Condition Loss}
\[
\mathcal{L}_b =
\|\bm{u}-\bm{u}_{BC}\|^2
+ \|T-T_{BC}\|^2
\]
\end{frame}

\begin{frame}{Why Adaptive Weighting Matters}
\begin{itemize}
\item Different PDEs have different stiffness
\item High-Ra momentum residuals dominate gradients
\item Learned weights improve conditioning
\end{itemize}
\end{frame}

%=====================================================

%=====================================================
\subsection{Training and Validation}

\begin{frame}{Training Data}
\begin{itemize}
\item No experimental or CFD data
\item Physics-only collocation points
\end{itemize}
\end{frame}

\begin{frame}{Reference Solution}
\begin{itemize}
\item COMSOL Multiphysics
\item Weak-form finite element solver
\end{itemize}
\end{frame}

\begin{frame}{Velocity Comparison}
\placeholder{COMSOL vs PINN Velocity Contours}
\end{frame}

\begin{frame}{Temperature Comparison}
\placeholder{COMSOL vs PINN Temperature Contours}
\end{frame}

\begin{frame}{Error Metrics}
\begin{itemize}
\item Relative $L_2$ norms
\item Improved PINN outperforms standard PINN
\end{itemize}
\end{frame}

%=====================================================


\begin{frame}{Work}
\begin{itemize}
\item Transient growth
\item Rotation and Marangoni effects
\item Real-time process optimization
\end{itemize}
\end{frame}




\begin{frame}
\Huge{\centerline{Thank You}}
\end{frame}

%----------------------------------------------------------------------------------------
\end{document} 
